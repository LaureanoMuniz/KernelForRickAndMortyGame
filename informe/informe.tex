% ******************************************************** %
%              TEMPLATE DE INFORME ORGA2 v0.1              %
% ******************************************************** %
% ******************************************************** %
%                                                          %
% ALGUNOS PAQUETES REQUERIDOS (EN UBUNTU):                 %
% ========================================
%                                                          %
% texlive-latex-base                                       %
% texlive-latex-recommended                                %
% texlive-fonts-recommended                                %
% texlive-latex-extra?                                     %
% texlive-lang-spanish (en ubuntu 13.10)                   %
% ******************************************************** %


\documentclass[a4paper]{article}
\usepackage[spanish]{babel}
\usepackage[utf8]{inputenc}
\usepackage{charter}   % tipografia
\usepackage{graphicx}
%\usepackage{makeidx}
\usepackage{paralist} %itemize inline
\usepackage{subcaption}


%\usepackage{float}
%\usepackage{amsmath, amsthm, amssymb}
%\usepackage{amsfonts}
%\usepackage{sectsty}
%\usepackage{charter}
%\usepackage{wrapfig}
%\usepackage{listings}
%\lstset{language=C}

% \setcounter{secnumdepth}{2}
\usepackage{underscore}
\usepackage{caratula}
\usepackage{url}
\usepackage{ragged2e}
\usepackage{hyperref}
\usepackage{pdfpages}


% ********************************************************* %
% ~~~~~~~~              Code snippets             ~~~~~~~~~ %
% ********************************************************* %

\usepackage{color} % para snipets de codigo coloreados
\usepackage{fancybox}  % para el sbox de los snipets de codigo

\definecolor{litegrey}{gray}{0.94}

\newenvironment{codesnippet}{%
	\begin{Sbox}\begin{minipage}{\textwidth}\sffamily\small}%
	{\end{minipage}\end{Sbox}%
		\begin{center}%
		\vspace{-0.4cm}\colorbox{litegrey}{\TheSbox}\end{center}\vspace{0.3cm}}



% ********************************************************* %
% ~~~~~~~~         Formato de las páginas         ~~~~~~~~~ %
% ********************************************************* %

\usepackage{fancyhdr}
\pagestyle{fancy}

%\renewcommand{\chaptermark}[1]{\markboth{#1}{}}
\renewcommand{\sectionmark}[1]{\markright{\thesection\ - #1}}

\fancyhf{}

\fancyhead[LO]{Sección \rightmark} % \thesection\ 
\fancyfoot[LO]{\small{Ivo Pajor, Laureano Muñiz, Luciana Gorosito}}
\fancyfoot[RO]{\thepage}
\renewcommand{\headrulewidth}{0.5pt}
\renewcommand{\footrulewidth}{0.5pt}
\setlength{\hoffset}{-0.8in}
\setlength{\textwidth}{16cm}
%\setlength{\hoffset}{-1.1cm}
%\setlength{\textwidth}{16cm}
\setlength{\headsep}{0.5cm}
\setlength{\textheight}{25cm}
\setlength{\voffset}{-0.7in}
\setlength{\headwidth}{\textwidth}
\setlength{\headheight}{13.1pt}

\renewcommand{\baselinestretch}{1.1}  % line spacing

% ******************************************************** %


\begin{document}


\thispagestyle{empty}
\materia{Organización del Computador II}
\submateria{Segundo Cuatrimestre de 2020}
\titulo{Trabajo Práctico III}
\subtitulo{System Programming}
\integrante{Ivo Pajor}{460/19}{ivo_pajor@hotmail.com}
\integrante{Laureano Muñiz}{498/19}{lau2000m@hotmail.com}
\integrante{Luciana Gorosito}{577/18}{lugorosito0@gmail.com}

\maketitle
\newpage

\thispagestyle{empty}
\vfill


\thispagestyle{empty}
\vspace{3cm}
\tableofcontents
\newpage


%\normalsize
\newpage

\section{Introducción}



\section{Desarrrollo}

\subsection{Ejercicio 1}

Para la realización de este ejercicio analizamos las estructuras \textbf{gdt_entry_t} y \textbf{gdt_descriptor_t} definidas por la cátedra. En esta implementación, la Tabla de Descriptores Globales(GDT) es un arreglo de  \textbf{gdt_entry_t} y su descriptor, que luego cargaremos en GDTR, es \textbf{gdt_descriptor_t}.\par
Siguiendo lo indicado en el primer item, definimos a partir del índice 10, 4 descriptores de segmento en la GDT utilizando la estructura \textbf{gdt_entry_t} atendiendo a las propiedades particulares de cada segmento. Puesto que estos segmentos deben direccionar los primeros 201 MB de memoria establecimos en todos el bit de G en 1, y definimos su base en 0x00000000h y su limite en 0x00C8FFh. Además, estos 4 son segmentos de código o datos de 32 bits por lo que el bit S se encuentra seteado en 1, el de D/B se encuentra seteado en 1 y el bit L en 0. El bit de DPL de cada uno está seteado en 0 o en 3 de acuerdo con el nivel de privilegio que le corresponda. Por último, los bits de tipo están seteados como 0xAh en caso de tratarse de un segmento de código y como 0x2h en caso de tratarse de un segmento de datos.\par
%%No se habla del bit P
%%Opcionalmente se puede poner una imagen con los bits de un descripto de segmento del manual.
Para poder pasar a modo protegido, en el kernel deshablitamos las interrupciones, cargamos en el registro GDTR la estructura \textbf{gdt_descriptor_t} y modificamos el ultimo bit del registro de control CR0, es decir, seteamos en 1 el bit de \textit{Protection Enable}. Posteriormente, escribimos el código necesario para saltar efectivamente a modo protegido. Debido a que este salto se consigue haciendo un \textit{far jump} a la próxima instrucción, designamos una etiqueta llamada \textbf{modo_protegido} a partir de la cual obtendremos el offset, mientras que como selector utilizamos el correspondiente al segmento de código de nivel 0.

\subsection{Ejercicio 2}
\subsection{Ejercicio 3}
\subsection{Ejercicio 4}
\subsection{Ejercicio 5}
\subsection{Ejercicio 6}
\subsection{Ejercicio 7}
\subsection{Ejercicio 8}
\subsection{Ejercicio 9}




\end{document}

